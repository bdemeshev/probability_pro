\documentclass[12pt]{article}

\usepackage{hyperref} % гиперссылки

\usepackage{tikz} % картинки в tikz
\usepackage{microtype} % свешивание пунктуации

\usepackage{array} % для столбцов фиксированной ширины

\usepackage{indentfirst} % отступ в первом параграфе

\usepackage{sectsty} % для центрирования названий частей
\allsectionsfont{\centering}

\usepackage{amsmath} % куча стандартных математических плюшек

\usepackage{comment} % добавление длинных комментариев

\usepackage[top=2cm, left=1.2cm, right=1.2cm, bottom=2cm]{geometry} % размер текста на странице

\usepackage{lastpage} % чтобы узнать номер последней страницы

\usepackage{enumitem} % дополнительные плюшки для списков
%  например \begin{enumerate}[resume] позволяет продолжить нумерацию в новом списке

\usepackage{caption} % что-то делает с подписями рисунков :)


\usepackage{fancyhdr} % весёлые колонтитулы
\pagestyle{fancy}
\lhead{Домашние работы для сдачи :)}
\chead{}
\rhead{2018-2019}
\lfoot{}
\cfoot{}
\rfoot{\thepage/\pageref{LastPage}}
\renewcommand{\headrulewidth}{0.4pt}
\renewcommand{\footrulewidth}{0.4pt}



\usepackage{todonotes} % для вставки в документ заметок о том, что осталось сделать
% \todo{Здесь надо коэффициенты исправить}
% \missingfigure{Здесь будет Последний день Помпеи}
% \listoftodos — печатает все поставленные \todo'шки



\usepackage{booktabs} % красивые таблицы
% заповеди из докупентации:
% 1. Не используйте вертикальные линни
% 2. Не используйте двойные линии
% 3. Единицы измерения - в шапку таблицы
% 4. Не сокращайте .1 вместо 0.1
% 5. Повторяющееся значение повторяйте, а не говорите "то же"



\usepackage{fontspec} % что-то про шрифты?
\usepackage{polyglossia} % русификация xelatex

\setmainlanguage{russian}
\setotherlanguages{english}

% download "Linux Libertine" fonts:
% http://www.linuxlibertine.org/index.php?id=91&L=1
\setmainfont{Linux Libertine O} % or Helvetica, Arial, Cambria
% why do we need \newfontfamily:
% http://tex.stackexchange.com/questions/91507/
\newfontfamily{\cyrillicfonttt}{Linux Libertine O}

\AddEnumerateCounter{\asbuk}{\russian@alph}{щ} % для списков с русскими буквами
\setlist[enumerate, 2]{label=\asbuk*),ref=\asbuk*}

%% эконометрические сокращения
\DeclareMathOperator{\Cov}{Cov}
\DeclareMathOperator{\Corr}{Corr}
\DeclareMathOperator{\Var}{Var}
\DeclareMathOperator{\E}{E}
\def \hb{\hat{\beta}}
\def \hs{\hat{\sigma}}
\def \htheta{\hat{\theta}}
\def \s{\sigma}
\def \hy{\hat{y}}
\def \hY{\hat{Y}}
\def \v1{\vec{1}}
\def \e{\varepsilon}
\def \he{\hat{\e}}
\def \z{z}
\def \hVar{\widehat{\Var}}
\def \hCorr{\widehat{\Corr}}
\def \hCov{\widehat{\Cov}}
\def \cN{\mathcal{N}}



\usepackage[bibencoding = auto,
backend = biber,
sorting = none,
style=alphabetic]{biblatex}

\addbibresource{em1_pset_v2.bib}



% делаем короче интервал в списках
\setlength{\itemsep}{0pt}
\setlength{\parskip}{0pt}
\setlength{\parsep}{0pt}


\begin{document}

\section{Общие правила}

\begin{enumerate}
  \item На один раунд выставляется четыре задачи.
  \item Из этих четырех задач нужно защитить две. Две задачи выбираются случайным образом ассистентом.
  \item Защищать задачи нужно без использования дополнительных материалов.
  \item За каждую задачу ставится ноль или один.
  \item Есть одна попытка поболтать с ассистентом.
  \item Арифметические ошибки несущественны.
  \item При решении можно гуглить и кооперироваться, а сдавать можно только в одиночку.
\end{enumerate}

\section{Раунд 1}

Срок сдачи: до 3 октября.

Принимают: Матвей Зехов, Аида Чушкина, Александр Реентович, Пётр Гармидер.

\begin{enumerate}

  \item Редкой болезнью более всего $0.01\%$ людей.

  Если человек болен, то имеющийся тест ошибочно признает его здоровым с вероятностью $5\%$
  и наоборот, если человек здоров, тест признает человека больным с вероятностью $5\%$.

  Судя по тесту, Васиссуалий болен. Какова вероятность того, что он действительно болен?

  % про биномиальные коэффициенты
  \item Игрок получает случайным образом 5 карт из колоды в 52 карты без джокеров.
  Найдите вероятность получения комбинаций:

  \begin{enumerate}
    \item Стрит-флеш. Пять карт одной масти подряд.
    \item Каре. Четыре карты одного достоинства и одна дополнительная карта.
    \item Фул-хаус. Три карты одного достоинства и две другие карты одного достоинства.
    \item Флеш. Пять карт одной масти. Исключая стрит-флеш.
    \item Трипс. Три карты одного достоинства и две дополнительные разные карты.
  \end{enumerate}

  Комментарий: здесь нужно защитить один случайно выбираемый пункт.



  % про метод первого шага
  \item Саша и Маша подкидывают монетку до тех пор, пока не выпадет
 последовательность РОО или РРО. Если игра закончится выпадением
 РОО, то выигрывает Саша, если РРО, то — Маша.
 \begin{enumerate}
 \item Какова вероятность выиграть у Маши?
 \item Чему равно среднее количество подбрасываний?
 \item Чему равно среднее количество решек?
\end{enumerate}


Комментарий: здесь нужно защитить один случайно выбираемый пункт,
при этом ассистент может указать любые комбинации из трёх орлов или решек для Саши и Маши.


  \item Есть три закрытых двери. За двумя из них — по козе, за третьей автомобиль.
  Вы выбираете одну из дверей. Допустим, Вы выбрали дверь А.
  Ведущий шоу, чтобы поддержать интригу, не открывает сразу выбранную Вами дверь.
  Сначала он открывает одну из дверей не выбранных Вами,
  причем ради интриги ведущий не открывает сразу и дверь с автомобилем.
  Допустим, ведущий открыл дверь B. И в этот момент он предлагает Вам изменить ваш выбор двери.

  Имеет ли смысл изменить свой выбор?

\end{enumerate}

\section{Раунд 2}



\end{document}
